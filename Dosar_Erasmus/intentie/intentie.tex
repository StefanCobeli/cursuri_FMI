%%%%%%%%%%%%%%%%%%%%%%%%%%%%%%%%%%%%%%%%%
% Thin Formal Letter
% LaTeX Template
% Version 1.11 (8/12/12)
%
% This template has been downloaded from:
% http://www.LaTeXTemplates.com
%
% Original author:
% WikiBooks (http://en.wikibooks.org/wiki/LaTeX/Letters)
%
% License:
% CC BY-NC-SA 3.0 (http://creativecommons.org/licenses/by-nc-sa/3.0/)
%
%%%%%%%%%%%%%%%%%%%%%%%%%%%%%%%%%%%%%%%%%

%----------------------------------------------------------------------------------------
%	DOCUMENT CONFIGURATIONS
%----------------------------------------------------------------------------------------

\documentclass[22]{letter}

% Adjust margins for aesthetics
\addtolength{\voffset}{-0.5in}
\addtolength{\hoffset}{-0.3in}
\addtolength{\textheight}{2cm}

%\longindentation=0pt % Un-commenting this line will push the closing "Sincerely," to the left of the page

%----------------------------------------------------------------------------------------
%	YOUR NAME & ADDRESS SECTION
%----------------------------------------------------------------------------------------

\signature{\c Stefan Cobeli} % Your name for the signature at the bottom

\date{\vspace{-5ex}}
%----------------------------------------------------------------------------------------
\newcommand\textbox[1]{%
  \parbox{.333\textwidth}{#1}%
}
\makeatletter
\newcommand*{\rom}[1]{\expandafter\@slowromancap\romannumeral #1@}
\makeatother
\title{Cover letter}
\begin{document}

%----------------------------------------------------------------------------------------
%	ADDRESSEE SECTION
%----------------------------------------------------------------------------------------

\begin{letter}{  } % Name/title of the addressee

%----------------------------------------------------------------------------------------
%	LETTER CONTENT SECTION
%----------------------------------------------------------------------------------------

\opening{\textit{ Faculty of Mathematics and Computer Science, University of Bucharest\\} }


\vspace{2\parskip}
\vspace{2\parskip}
\textbf{Letter of motivation, ERASMUS departure : } \textbf{\c Stefan Cobeli}
\vspace{2\parskip} % Extra whitespace for aesthetics
\vspace{2\parskip} % Extra whitespace for aesthetics

I'm a student of the\textit{ Mathematics} department of our faculty, in the third year of the specialization \textit{Mathematics and Computer Science}  and in the next year I want to follow the last year of the \textit{Computer Science} license. 


I think that it's important for a student to gain as much information as he can from any field, because the real life problems aren't algorithmic and usually their solving is implied by a combination of  skills and knowledge from miscellaneous areas. For this reason, and  also because I like it, I attend to courses outside our faculty and to less formal activities.

The same domain has a lot of sides and the approach of it can vary a lot from one view, to another. I'm curious to discover how is\textit{ Computer Science} taught in other countries and I want to share my experience after the end of it with my younger colleagues, and with our faculty in order to develop further our quality.
Another reason for which I want to visit a foreign faculty is that I like to learn how people think and this will be a challenge for me outside Romania.

After I did some research about the colleges from the list of possible destinations, I made a list of places I want to go. My first choice would be\\ \textit{Freie Universit\" at Berlin}, because it's a prestigious university and has a good faculty of \textit{Computer Science} where I would like to attend to the courses of Bioinformatics, Software Engineering and Computer Security, courses that are also studied in our faculty. I will renew my course plan depending on what my advisor will say. The one that I've made it's only a template.

  Also, I found interesting the universities \textit{Instituto Superior Miguel Torga} and \textit{Universidade da Coru\~ na}, because the local language is very close to the \\Romanian language and I could more easily interact with those around me.
  
  In conclusion I think that ERASMUS is a good opportunity for me to develop my communication and learn about different styles of \textit{Computer Science} teaching approaches.
  
  
%----------------------------------------------------------------------------------------

\end{letter}
 
 
\end{document}