%%%%%%%%%%%%%%%%%%%%%%%%%%%%%%%%%%%%%%%%%
% Thin Sectioned Essay
% LaTeX Template
% Version 1.0 (3/8/13)
%
% This template has been downloaded from:
% http://www.LaTeXTemplates.com
%
% Original Author:
% Nicolas Diaz (nsdiaz@uc.cl) with extensive modifications by:
% Vel (vel@latextemplates.com)
%
% License:
% CC BY-NC-SA 3.0 (http://creativecommons.org/licenses/by-nc-sa/3.0/)
%
%%%%%%%%%%%%%%%%%%%%%%%%%%%%%%%%%%%%%%%%%

%----------------------------------------------------------------------------------------
%	PACKAGES AND OTHER DOCUMENT CONFIGURATIONS
%----------------------------------------------------------------------------------------

\documentclass[a4paper, 11pt]{article} % Font size (can be 10pt, 11pt or 12pt) and paper size (remove a4paper for US letter paper)

\usepackage[protrusion=true,expansion=true]{microtype} % Better typography
\usepackage{graphicx} % Required for including pictures
\usepackage{wrapfig} % Allows in-line images

\usepackage{mathpazo} % Use the Palatino font
\usepackage[T1]{fontenc} % Required for accented characters
\linespread{1.05} % Change line spacing here, Palatino benefits from a slight increase by default

\makeatletter
\renewcommand\@biblabel[1]{\textbf{#1.}} % Change the square brackets for each bibliography item from '[1]' to '1.'
\renewcommand{\@listI}{\itemsep=0pt} % Reduce the space between items in the itemize and enumerate environments and the bibliography

\renewcommand{\maketitle}{ % Customize the title - do not edit title and author name here, see the TITLE block below
\begin{flushright} % Right align
{\LARGE\@title} % Increase the font size of the title

\vspace{50pt} % Some vertical space between the title and author name

{\large\@author} % Author name
\\\@date % Date

\vspace{40pt} % Some vertical space between the author block and abstract
\end{flushright}
}

%----------------------------------------------------------------------------------------
%	TITLE
%----------------------------------------------------------------------------------------

\title{\textbf{Single Bacteria type grows 60\% better on Space than on Earth}\\ % Title
} % Subtitle

\author{\textsc{ Cobeli \c Stefan} % Author
\\{\textit{College of Mathematics and Computer Science, University of Bucharest }}} % Institution

\date{\today} % Date

%----------------------------------------------------------------------------------------

\begin{document}

\maketitle % Print the title section

%----------------------------------------------------------------------------------------
%	ABSTRACT AND KEYWORDS
%----------------------------------------------------------------------------------------

%\renewcommand{\abstractname}{Summary} % Uncomment to change the name of the abstract to something else

\begin{abstract}
 Researchers from University of California 
 had the idea to collect simple bacteria from different
 environments on Earth and compare their behaviour on normal
 environment with their behaviour on space.
 The starting problem was that until now the attention
 has been focused on the microbes that have potential
 pathogenic risk and nobody studied the ordinary bacteria on space .
The result was surprising. Of the 48 types of bacteria
 sent on ISS all of them had similar behaviour, with
 one exception. Bacillus safensis grew 60\% better on ISS than on Earth.
\end{abstract}

\hspace*{3,6mm} % Keywords

\vspace{30pt} % Some vertical space between the abstract and first section

%----------------------------------------------------------------------------------------
%	ESSAY BODY
%----------------------------------------------------------------------------------------

\section*{Summary}

To accomplish this task, it was made a nationwide
 campain to popularise  science, named Project MERCCURI
 (Microbial Ecology Research Combining Citizen 
and University Researchers on the ISS). 
The people had to collect microbes from public places
 like museums, sports events,  schools, pools and send
 them to researchers and the most interesting microbes
 were selected. The bacteria winners were sent to ISS.
 
 
  After the selection, the researchers send
 samples with the winners on Dragon spacecraft, which was 
 launched on April 18th 2014.
The result was that Bacillus safensis bacteria grew 60\%
 better on the board of ISS than on Earth. 
 The reason for this is still a mystery. 
Clues may be found on the genome sequencing of the bacteria,
 that was recently determined .


\section*{Conclusion}
The lead author of the article Dr. David Coll said that:
 "Understanding how microbes behave in microgravity is critically 
 important for planning long-term manned space flight but also
 has the possibility of providing new insights into how these
 microbes behave in human constructed environments on Earth."
Also this study engaged the public in the world of 
science and research. The authors hope that this article will
 be a start of inspiration for young and adults to get involved 
 more in science world.




%----------------------------------------------------------------------------------------
%	BIBLIOGRAPHY
%----------------------------------------------------------------------------------------

\bibliographystyle{unsrt}

\bibliography{sample}

%----------------------------------------------------------------------------------------

\end{document}