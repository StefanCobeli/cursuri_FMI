%%%%%%%%%%%%%%%%%%%%%%%%%%%%%%%%%%%%%%%%%
% University/School Laboratory Report
% LaTeX Template
% Version 3.1 (25/3/14)
%
% This template has been downloaded from:
% http://www.LaTeXTemplates.com
%
% Original author:
% Linux and Unix Users Group at Virginia Tech Wiki 
% (https://vtluug.org/wiki/Example_LaTeX_chem_lab_report)
%
% License:
% CC BY-NC-SA 3.0 (http://creativecommons.org/licenses/by-nc-sa/3.0/)
%
%%%%%%%%%%%%%%%%%%%%%%%%%%%%%%%%%%%%%%%%%

%----------------------------------------------------------------------------------------
%	PACKAGES AND DOCUMENT CONFIGURATIONS
%----------------------------------------------------------------------------------------

\documentclass{article}

\usepackage[version=3]{mhchem} % Package for chemical equation typesetting
\usepackage{siunitx} % Provides the \SI{}{} and \si{} command for typesetting SI units
\usepackage{graphicx} % Required for the inclusion of images
\usepackage{natbib} % Required to change bibliography style to APA
\usepackage{amsmath} % Required for some math elements 
\usepackage[yyyymmdd]{datetime}

\setlength\parindent{0pt} % Removes all indentation from paragraphs

\renewcommand{\labelenumi}{\alph{enumi}.} % Make numbering in the enumerate environment by letter rather than number (e.g. section 6)

%\usepackage{times} % Uncomment to use the Times New Roman font

%----------------------------------------------------------------------------------------
%	DOCUMENT INFORMATION
%----------------------------------------------------------------------------------------

\title{\textbf{Studiu de caz } \\Conflicte \^ in \c scoal\u a \\  } % Title

\author{M\^indril\u a Claudiu(gr. 301)} % Author name

%\date{\today} % Date for the report
\def\mydate{\leavevmode\hbox{\the\year-\twodigits\month-\twodigits\day}}
\def\twodigits#1{\ifnum#1<10 0\fi\the#1}

\begin{document}

\maketitle % Insert the title, author and date


% If you wish to include an abstract, uncomment the lines below

%----------------------------------------------------------------------------------------
%	SECTION 1
%----------------------------------------------------------------------------------------

\section{Problem\u a}
La ora de matematic\u a, dup\u a sus\c tinerea unui test \c si aducerea rezultatelor, un elev este nemul\c tumit de nota sa.


\label{definitions}
%----------------------------------------------------------------------------------------
%	SECTION 2
%----------------------------------------------------------------------------------------

\section{Cauze}
Factorii care pot determina acest eveniment ar fi:

\begin{enumerate}

\item Lipsa de aten\c tie \^ in timpul corect\u arii testelor.
\item Predearea defectuoas\u a subiectului examinat de c\u atre profesor.
\item Preg\u atirea insuficient\u a elevului \^ in cauz\u a.


\end{enumerate}



%----------------------------------------------------------------------------------------
%	SECTION 3
%----------------------------------------------------------------------------------------

\section{Solu\c tie}

Situa\c tia de fapt se poate rezolva prin purtarea unei discu\c tii \^ in particular cu elevul respectiv \c si reanalizarea lucr\u arii \^ impreun\u a cu el.

%----------------------------------------------------------------------------------------
%	SECTION 4
%----------------------------------------------------------------------------------------

\section{Rezultat}

Obiectivul vizat prin aplicarea acestei metode este con\c stientizarea elevelui asupra felului \^ in care i s-a acordat nota \c si corectarea eventualelor erori de notare.\\
\^ In cazul \^ in care mai mul\c ti elevi au o problem\u a similar\u a, trebuie luat\u a \^ in \\
considerare o schimbare a metodei de predare a subiectului respectiv.\\

%----------------------------------------------------------------------------------------
%	SECTION 5
%----------------------------------------------------------------------------------------

\section{Conflicte posibile}

\^ In cazul unor abateri de la planul de mai sus, din diverse motive\\(neaten\c tie, neglijen\c t\u a sau pur \c si simplu o situa\c tie neprev\u azut\u a) pot ap\u area divergen\c te \^ intre opiniile profesorului \c si cele ale elevului vizat \c si astfel poate lua na\c stere un conflict nedorit. \\ Situa\c tia se rezolv\u a \^ intr-un mod calm \c si c\^ at se poate de echilibrat, astfel \^ incat restul clasei s\u a fie c\^ at mai pu\c tin afectat\u a de apari\c tia acestui eveniment.\\
Este recomandat ca fiecare caz s\u a fie tratat \^ in mod diferit, deoarece\\ orice elev are \^ insu\c sirile sale unice \c si astfel comportamentul fiec\u aruia poate varia de la caz, la caz. Pentru aceasta, profesorul trebuie s\u a apeleze la cuno\c stin\c tele sale didactice, pedagogice sau la inteligen\c ta sa social\u a, dar este foarte posibil ca situa\c tia s\u a fie solu\c tionat\u a \c si prin spontaneitate.
\end{document}