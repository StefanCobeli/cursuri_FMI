%%%%%%%%%%%%%%%%%%%%%%%%%%%%%%%%%%%%%%%%%
% University/School Laboratory Report
% LaTeX Template
% Version 3.1 (25/3/14)
%
% This template has been downloaded from:
% http://www.LaTeXTemplates.com
%
% Original author:
% Linux and Unix Users Group at Virginia Tech Wiki 
% (https://vtluug.org/wiki/Example_LaTeX_chem_lab_report)
%
% License:
% CC BY-NC-SA 3.0 (http://creativecommons.org/licenses/by-nc-sa/3.0/)
%
%%%%%%%%%%%%%%%%%%%%%%%%%%%%%%%%%%%%%%%%%

%----------------------------------------------------------------------------------------
%	PACKAGES AND DOCUMENT CONFIGURATIONS
%----------------------------------------------------------------------------------------

\documentclass{article}

\usepackage[version=3]{mhchem} % Package for chemical equation typesetting
\usepackage{siunitx} % Provides the \SI{}{} and \si{} command for typesetting SI units
\usepackage{graphicx} % Required for the inclusion of images
\usepackage{natbib} % Required to change bibliography style to APA
\usepackage{amsmath} % Required for some math elements 
\usepackage[yyyymmdd]{datetime}

\setlength\parindent{0pt} % Removes all indentation from paragraphs

\renewcommand{\labelenumi}{\alph{enumi}.} % Make numbering in the enumerate environment by letter rather than number (e.g. section 6)

%\usepackage{times} % Uncomment to use the Times New Roman font

%----------------------------------------------------------------------------------------
%	DOCUMENT INFORMATION
%----------------------------------------------------------------------------------------

\title{\textbf{Programare Procedural\u a (Func\c tii) } \\ Plan de lec\c tie \\ \textit{ Clasa a  XI -a} } % Title

\author{\c Stefan \textsc{Cobeli}} % Author name

%\date{\today} % Date for the report
\def\mydate{\leavevmode\hbox{\the\year-\twodigits\month-\twodigits\day}}
\def\twodigits#1{\ifnum#1<10 0\fi\the#1}

\begin{document}

\maketitle % Insert the title, author and date


% If you wish to include an abstract, uncomment the lines below

%----------------------------------------------------------------------------------------
%	SECTION 1
%----------------------------------------------------------------------------------------
Timpul necesar lec\c tiei este de 3 ore ( 150 de minute ).
\section{Obiective}

\begin{itemize}

\item Prezentare paradigmei \textit{Program\u arii procedurale} \c si amintirea unor limbaje bazate pe aceast\u a paradigm\u a.

\item Prezentarea diferen\c telor dintre aceast\u a paradigm\u a \c si alte paradigme de programare existente.

\item R\u aspunderea la \^ intrebarea: \textit{ De ce este \textbf{C} un limbaj procedural?}

\item Prezentarea Sintaxei func\c tiilor.

\item Explicarea semnifica\c tiei transmiterii parametrilor c\u atre o func\c tie\\ ( transferul prin valoare ).

\item Explicarea dezavantajelor transferului prin valoare \c si metode de solu\c tionare ale unor probleme care pot ap\u area ( transferul prin adres\u a ).

\item Prezentarea unor scurte informa\c tii despre pointeri .\\ De ce sunt utili \c si ce probleme pot rezolva.


\end{itemize}
\label{definitions}
%----------------------------------------------------------------------------------------
%	SECTION 2
%----------------------------------------------------------------------------------------

\section{Exerci\c tii propuse}

\begin{enumerate}

\item Dup\u a \^ indeplinirea primelor trei obiective, propunem realizarea\\ unui program simplu, fiec\u arui elev.\\ Spre exemplu, calcularea sumei numerelor p\^ an\u a la o anumit\u a valoare sau concatenarea a dou\u a \textit{String}-uri (\textit{O list\u a de probleme asem\u an\u atoare urmeaz\u a a fi realizat\u a}).

\item Dup\u a prezentarea sintaxei func\c tiilor, propunem elevilor s\u a modifice \\programul realizat anterior, astfel \^ inc\^ at cerin\c ta programului s\u a fie realizat\u a cu ajutorul unei func\c tii.

\item Realizarea unui program care interschimb\u a valorile dintre dou\u a variabile de tip \^ intreg, f\u ar\u a func\c tii.

\item Crearea unei func\c tii care s\u a realizeze cerin\c ta anterioar\u a.

\end{enumerate}



%----------------------------------------------------------------------------------------
%	SECTION 3
%----------------------------------------------------------------------------------------

\section{Concluzii}

\paragraph{}
Repetarea no\c tiunilor de \textit{Programare procedural\u  a}, \textit{func\c tie}, \textit{pointer}.


%----------------------------------------------------------------------------------------
%	SECTION 4
%----------------------------------------------------------------------------------------

\section{Evaluarea elevilor \c si tema pentru acas\u a }

\paragraph{}
Pe parcursul pred\u arii se vor lansa propuneri de probleme interesante \c si rezolvarea lor. Unele probleme vor fi propuse pentru studiu particular. 

%----------------------------------------------------------------------------------------
%	SECTION 5
%----------------------------------------------------------------------------------------

\section{Materiale folosite pentru preg\u atirea lec\c tiei}

\begin{enumerate}
\item Manualul \c scolar de informatic\u a.
\item Exemple interesante care apar frecvent ( se va folosi internetul \c si alte mijloace auxiliare pentru g\u asirea lor ).
\item Tutoriale de informatic\u a.
\end{enumerate}

\end{document}