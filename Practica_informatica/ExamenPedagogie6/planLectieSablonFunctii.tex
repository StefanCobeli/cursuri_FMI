%%%%%%%%%%%%%%%%%%%%%%%%%%%%%%%%%%%%%%%%%
% University/School Laboratory Report
% LaTeX Template
% Version 3.1 (25/3/14)
%
% This template has been downloaded from:
% http://www.LaTeXTemplates.com
%
% Original author:
% Linux and Unix Users Group at Virginia Tech Wiki 
% (https://vtluug.org/wiki/Example_LaTeX_chem_lab_report)
%
% License:
% CC BY-NC-SA 3.0 (http://creativecommons.org/licenses/by-nc-sa/3.0/)
%
%%%%%%%%%%%%%%%%%%%%%%%%%%%%%%%%%%%%%%%%%

%----------------------------------------------------------------------------------------
%	PACKAGES AND DOCUMENT CONFIGURATIONS
%----------------------------------------------------------------------------------------

\documentclass{article}

\usepackage{multicol}
\usepackage{subfigure}
\usepackage[utf8x]{inputenc}
\usepackage[romanian]{babel}
\usepackage[version=3]{mhchem} % Package for chemical equation typesetting
\usepackage{siunitx} % Provides the \SI{}{} and \si{} command for typesetting SI units
\usepackage{graphicx} % Required for the inclusion of images
\usepackage{natbib} % Required to change bibliography style to APA
\usepackage{amsmath} % Required for some math elements 
\usepackage[yyyymmdd]{datetime}
\usepackage{enumitem}
\setlength\parindent{0pt} % Removes all indentation from paragraphs

\renewcommand{\labelenumi}{\alph{enumi}.} % Make numbering in the enumerate environment by letter rather than number (e.g. section 6)

%\usepackage{times} % Uncomment to use the Times New Roman font

%----------------------------------------------------------------------------------------
%	DOCUMENT INFORMATION
%----------------------------------------------------------------------------------------

\title{\textbf{Programare Procedural\u a (Func\c tii) } \\ Plan de lec\c tie \\ \textit{ Clasa a  XI -a} } % Title

\author{\c Stefan \textsc{Cobeli}} % Author name

%\date{\today} % Date for the report
\def\mydate{\leavevmode\hbox{\the\year-\twodigits\month-\twodigits\day}}
\def\twodigits#1{\ifnum#1<10 0\fi\the#1}
\newcommand{\itl}{\textit}
\begin{document}

\maketitle % Insert the title, author and date


% If you wish to include an abstract, uncomment the lines below

%----------------------------------------------------------------------------------------
%	SECTION 1
%----------------------------------------------------------------------------------------
Timpul necesar lec\c tiei este de 3 ore ( 150 de minute ).
\section{Organizarea}

	\subsection{Organizarea administrativă:}
		\begin{itemize}[label={-}]
			\item Disciplina: \itl{Informatică};
			\item Unitatea de învățământ: \textit{Colegiul Național Gheorghe Șincai};
			\item Nivelul de învățământ: \itl{Clasa a XI-a};
			\item Sală necesară: \itl{Laborator de Informatică};
			\item Timpul necesar predării: \itl{150 de minute};
				
		\end{itemize}				
		
		
		
	\subsection{Organizarea pedagogică:}
		\begin{itemize}[label={-}]
			\item Subiectul abordat: \itl{Programarea Procedurală};
			\item Organizarea sălii: Elevii vor folosii calculatoarele din laborator. 
			\item Tipul lecției: Lecție de predare.
		\end{itemize}
%%%%%%%%%%%%%%%%%%%%%%%%%%%%%%
\newpage
\section{Planificarea lecției}
	\subsection{Scopul general:}
		\begin{itemize}[label={-}]
			
			\item Prezentare paradigmei \textit{Program\u arii procedurale} \c si amintirea unor limbaje bazate pe aceast\u a paradigm\u a.

			\item Prezentarea diferen\c telor dintre aceast\u a paradigm\u a \c si alte paradigme de programare existente.

			\item R\u aspunderea la \^ intrebarea: \textit{ De ce este \textbf{C} un limbaj procedural?}

			\item Prezentarea Sintaxei func\c tiilor.

			\item Explicarea semnifica\c tiei transmiterii parametrilor c\u atre o func\c tie\\ ( transferul prin valoare ).

			\item Explicarea dezavantajelor transferului prin valoare \c si metode de solu\c tionare ale unor probleme care pot ap\u area ( transferul prin adres\u a ).

			\item Prezentarea unor scurte informa\c tii despre pointeri .\\ De ce sunt utili \c si ce probleme pot rezolva.


		\end{itemize}
	\subsection{Obiective operaționale:}
		\paragraph{}Elevii, după terminarea predării, ar trebui să fie capabili să recunască sintaxa unei funcții, să alcătuiască funcții simple și să 
cunoască importanța structurării programelor în proceduri.

	\subsection{Conținutul de bază:}
		\paragraph{}Înțelegerea importanței structurării unui program în bucăți mici și simplu de înțeles, dar, de asemenea foarte importantă este înțelegerea sintaxei, care nu este complicată, dar este indispensabilă.
	
	\subsection{Metode și Procedee de predare:}
	\begin{multicols}{2}
		\subparagraph{i. Metode:}
			\begin{itemize}[label={-}]
				\item Conversația;
				\item Explicația;
				\item Învățarea prin descoperire;
				\item Exercițiul;
				\item Activitatea independentă.
			\end{itemize}
		\columnbreak
		\subparagraph{ii. Procedee:}
			\begin{itemize}[label={-}]
				\item Manualul;
				\item Materialele de pe internet;
				\item Tutoriale;
				\item Probleme interesante.
				\item[]
			\end{itemize}
	\end{multicols}
%%%%%%%%%%%%%%%%%%%%%
	\subsection{Evaluarea:}
		\begin{itemize}
			\item[-] Inițială: Se va realiza prin conversație;
			\item[-] Continuă: Se va realiza prin evaluarea lucrului individual;
			\item[-] Finală: Se va realiza pe baza impresiei generale creeate de către elevi.
		\end{itemize}
		
%%%%%%%%%%%%%%%%%%%%%%%%%%%%%%%%%%

\section{Realizarea și cuprinsul lecției}
	\subsection{Moment organizatoric:\\ \itl{10 minute}}
		\paragraph{}În cadrul acestuia se vor explica pe scurt obiectivele și vor fi menționate materialele care vor servi la buna desfășurare a predării.
		
	\subsection{Comunicarea scopului general și a obiectivelor operaționale: \itl{5 minute}}
		\paragraph{}Conținutul acestei secțiuni se întrepătrunde cu cel al precedentei. În plus, se mai pot preciza metodele de evaluare (e.g. rezzolvarea problemelor suplimentare). 	
	
	\subsection{Evaluarea inițială:}
		\paragraph{} Deoarece este o lecție de predare, nu vom avea parte de o evaluare inițială. Dacă profesorul  dorește să recapituleze subiecte abordate în trecut, se pate gândi la o evaluare restrânsă. Lăsăm această decizie în mâinile didactului în cauză.
	
	\subsection{Predare, Învățare și Evaluare continuă:\\ \itl{110 minute}}
	\paragraph{} În continuare vom prezenta câteva sugestii pentru teme și probleme ce ar trebui abordate pe parcursul orei.	
	\begin{enumerate}

		\item Dup\u a \^ indeplinirea primelor trei obiective, propunem realizarea\\ unui program simplu, fiec\u arui elev.\\ Spre exemplu, calcularea sumei numerelor p\^ an\u a la o anumit\u a valoare sau concatenarea a dou\u a \textit{String}-uri (\textit{O list\u a de probleme asem\u an\u atoare urmeaz\u a a fi realizat\u a}).

		\item Dup\u a prezentarea sintaxei func\c tiilor, propunem elevilor s\u a modifice \\programul realizat anterior, astfel \^ inc\^ at cerin\c ta programului s\u a fie realizat\u a cu ajutorul unei func\c tii.

		\item Realizarea unui program care interschimb\u a valorile dintre dou\u a variabile de tip \^ intreg, f\u ar\u a func\c tii.

		\item Crearea unei func\c tii care s\u a realizeze cerin\c ta anterioar\u a.

\end{enumerate}


\subsection{Concluzii:\\ \itl{10 minute}}

\paragraph{}
Repetarea no\c tiunilor de \textit{Programare procedural\u  a}, \textit{func\c tie}, \textit{pointer}.\\În cadrul acestei secțiui ne vom ocupa și de partea de \itl{feedback} pe care elevii o vor avea la dispoziție.


%----------------------------------------------------------------------------------------
%	SECTION 4
%----------------------------------------------------------------------------------------

\subsection{Evaluarea elevilor \c si tema pentru acas\u a: \\ 10 minute }

\paragraph{}
Pe parcursul pred\u arii se vor lansa propuneri de probleme interesante \c si rezolvarea lor. Unele probleme vor fi propuse pentru studiu particular. 

%----------------------------------------------------------------------------------------
%	SECTION 5
%----------------------------------------------------------------------------------------
\hfill

\hrulefill
\hrule
\hrulefill
\begin{flushright}
	Pe parcursul întregii predări, metoda didactică preponderent folosită va fi \itl{Comunicarea}.

\end{flushright}
\end{document}
